Greedy algorithms are an informal grouping of algorithms characterized by the following property: We have to make a number of choices until we are done, and at each step we choose the most attractive option.

A greedy algorithm emphasizes local optimization over global optimization: at each step it makes the optimal local choice.
That may or may not yield a good result overall.
For example, eating the cheapest available food every day saves money in the short run (locally optimal) but is not healthy in the long run (globally optimal).

But greedy algorithms are easy to implement and usually very efficient: making a locally optimal choice is usually much easier than making a globally optimal choice.
For example, finding the cheapest available food just requires browsing items in the market.
But finding out what food is healthy in the long run may require extensive research.
Humans have a strong tendency towards making local choices because they require so much fewer mental effort and yield immediate gratification.

\section{General Structure}

Consider a set $M$ with a weight function $w:M\to \N$ (or any other other set of positive numbers).
For a set $S\sq M$, we define the weight of $S$ by summing the weights of the elements.
 \[w(S)=\Sigma_{x\in S} w(x)\]

Moreover, consider a property $Acceptable$, i.e., if $S\sq M$, then $Acceptable(S)$ is a boolean.

Our goal is to find an acceptable subset of $M$ with maximal weight.

The generic greedy algorithm proceeds as follows:
\begin{acode}
\acomment{precondition: $elements$ is the list of all elements of $M$ sorted decreasingly by $w$}\\
\afun{greedy[M]}{elements:List[M], Acceptable: Set[M]\to\Bool}{
  solution := \anew{Set[M]}{}\\
  foreach(elements, \alam{x}{\aifI{Acceptable(solution \cup \{x\})}{insert(solution,x)}})\\
  solution
}
\end{acode}
Here $solution \cup \{x\}$ is an immutable operation that returns a new set, and $insert(solution,x)$ is a mutable operation that changes $solution$.

We can also use the function $greedy$ to find an acceptable subset with \emph{minimal} weight: we simply sort $elements$ by \emph{increasing} weight.

\section{Matroids}

To understand when the generic greedy algorithm yields an optimal results, we introduce matroids:

\begin{definition}[Matroid]
A \textbf{matroid} consists of
\begin{compactitem}
\item a finite set $M$
\item a property $Acceptable:Set[M]\to\B$ (subsets with this property are called \textbf{acceptable}\footnotemark)
\end{compactitem}
such that the following holds
\begin{compactitem}
 \item $M$ has at least one acceptable subset.
 \item $Acceptable$ is subset-closed, i.e., subsets of acceptable sets are also acceptable.
 \item If $A$ and $B$ are acceptable and $|A|>|B|$, then $B$ can be increased to an acceptable set $B\cup\{x\}$ by adding an element $x\in A$ to $B$. (Thus, we must have $x\in A\sm B$.)
\end{compactitem}
\end{definition}
\footnotetext{The literature calls them \emph{independent}, but \emph{acceptable} is a more intuitive name for greedy algorithms.}

The third property is the critical one: it guarantees that it does not matter which elements we add to an acceptable set, we always eventually get an acceptable set of maximal size.
Thus, local choices (which element to add) can never lead to a dead end.
More formally, we can state this as follows:
\begin{theorem}
We call an acceptable set $S$ that has no acceptable superset $S'\supset S$ a \textbf{base}.

Then, in a matroid, all bases have the same size.
\end{theorem}
\begin{proof}
Exercise.
\end{proof}

\section{Greedy Algorithms for Matroids}

\paragraph{Correctness}
Finding a base is always easy: start with the empty set and keep adding elements as long as the resulting set remains acceptable.

Now the matroid property guarantees that all bases have the same size.
Thus, it does not matter which elements we add---eventually we get an acceptable set of maximal size.

If we want to find not only an acceptable set of maximal \emph{size} but an acceptable set of maximal \emph{weight}, we simply add the elements in order of weight---that is exactly what the greedy algorithm does.
Formally, we have:

\begin{theorem}
If $M$ and $Acceptable$ form a matroid, then the greedy algorithm finds a base with greatest possible weight.
\end{theorem}

The corresponding theorem holds for finding the base with smallest possible weight.

\paragraph{Complexity}
The main structure of the greedy algorithm is linear in the number $|M|$ of elements.
But we also have to sort the elements once and check $Acceptable$ at every step.
We know sorting takes $\Theta(|M|\log |M|)$.
So if we can check $Acceptable(S)$ in $O(\log|S|)$, the overall run time (including sorting) is in $O(|M|\log|M|)$.

To be efficient, we usually implement the acceptability check in a greedy algorithm slightly smarter:
\begin{acode}
\aclass{Solution[M]}{}{}{
  \afun{insert}{m:M}{\ldots}\\
  \afun{acceptableWith}{m:M}{\ldots}
}\\
\\
\acomment{precondition: $elements$ is the list of all elements of $M$ sorted decreasingly by $w$}\\
\afun{greedy[M]}{elements:List[M]}{
  solution := \anew{Solution[M]}{}\\
  foreach(elements, \alam{x}{\aifI{solution.acceptableWith(x)}{solution.insert(x)}})\\
  solution
}
\end{acode}

Here $S.acceptableWith(x)$ is specified as follows:
\begin{compactitem}
 \item precondition: $Acceptable(S)$
 \item postcondition: if $S.acceptableWith(x)$, then $Acceptable(S\cup\{x\})$
\end{compactitem}
This works because the greedy algorithm only every needs to check $Acceptable(S\cup\{x\})$ and only if $S$ is already known to be acceptable.

$S.acceptableWith(x)$ can often be implemented much faster than $Acceptable(S\cup\{x\})$ because:
\begin{compactitem}
 \item We do not have to copy the set $S$ to build $S\cup\{x\}$.
 \item The acceptability check can use the information that $S$ is already acceptable.
\end{compactitem}

\section{Examples}

\subsection{Kruskal's Algorithm}

Kruskal's algorithm from Sect.~\ref{sec:ad:spanningtree} is a simple example of a greedy algorithm.

\paragraph{Correctness}
To show that it is correct, we only have to show that it operates on a matroid.

We use the following matroid:
\begin{compactitem}
 \item The set $M$ is the set of edges of the graph $G=(N,E)$.
 \item A set $S\sq E$ is acceptable if the graph $(N,S)$ is a set of trees.
\end{compactitem}

We have to prove the matroid properties:
\begin{compactitem}
 \item There is an acceptable set. For example, $(N,\es)$ is a graph where every node is a tree by itself.
 \item Subsets of acceptable sets are acceptable. Taking an edge away from a tree splits it into two trees. Thus, removing edges from a set of trees again yields a set of trees.
 \item For the critical third property, assume that $(N,A)$ and $(N,B)$ are sets of trees such that $A$ contains more edges than $B$.
  We have to find an edge $x\in A\sm B$ that we can add to $B$.
  We pick any $x\in A$ that connects two nodes that are not in the same tree in $(N,B)$.
  Then $(N,B\cup\{x\})$ is a set of trees again.
  We only have to check that such an $x$ exists: If there were no such $x$, the trees in $A$ and $B$ would consist of the same nodes; but then $A$ cannot have more edges then $B$ because the number of edges in a tree is already fixed by the number of nodes.
\end{compactitem}

Thus, we immediately know that Kruskal's algorithm is correct.

\paragraph{Complexity}
We have $Acceptable=isSetOfTrees$.

To improve efficiency, we implement an appropriate data structure for the class $Solution$.
This is indeed possible in $O(\log|S|)$.
Then we obtain $\Theta(|E|\log|E|)$ as the overall run time of Kruskal's algorithm.

The idea behind the implementation is that $S.acceptableWith(x)$ only has to check that $x$ connects two nodes from different trees.
By cleverly storing the set of trees, we can check that in $O(\log|S|)$.

\subsection{Dijkstra's Algorithm}

Because the term \emph{greedy algorithm} is not defined precisely, not every algorithm that has a greedy flavor is a special case of the matroid algorithm.

A counter-example is Dijkstra's algorithm from Sect.~\ref{sec:ad:shortestpath}.


%\subsection{Greatest Flow}
%
%Consider the greatest flow problem from Sect.~\ref{sec:ad:maximalflow}.
%We want to construct a greedy algorithm for it by identifying an appropriate matroid structure.
%
%For a path $p$ from $source$ to $sink$, we write $Cap(p)$ for the capacity of $p$.
%Recall that the capacity of a path is the minimum of the weights of the edges of $p$.
%
%We want to describe a flow as a set $S$ of such paths.
%Along each path of $S$, we try to flow as much as the capacity of the path.
%But we have to make sure we use each edge $e$ only up to its capacity $w(e)$.
%Because $e$ may be part of multiple paths $p\in S$, the sum of the capacities of these paths $p$ must stay below $w(e)$.
%
%That leads us to using the following matroid:
%\begin{compactitem}
%  \item $M$ is the set of paths from the source to the sink.
%  \item A set $S$ of paths is acceptable if for every edge $e\in E$
%   \[\Sigma_{p\in S\mwith e\in p} Cap(p)\leq w(e)\]
%\end{compactitem}
%Then every acceptable set yields a flow.
%
%We have to prove the matroid properties:
%\begin{compactitem}
%  \item There is an acceptable set, e.g., the empty set.
%  \item Subsets of acceptable sets are acceptable because lowering the flow cannot introduce a violation of the capacity conditions.
%  \item Consider two sets $A$ and $B$ of paths such that $|A|>|B|$.
%  
%\end{compactitem}