\section{Overview}

A divide-and-conquer algorithm consists of $3$ steps:
\begin{compactenum}
 \item the input is divided into parts,
 \item each part is processed recursively,
 \item the results of the parts are aggregate into the result of the whole.
\end{compactenum}
Divide-and-conquer works whenever a problem of size $n$ can be reduced to multiple smaller subproblems of the same kind.

The simplest special case arises when working on a list that is divided into two parts.
$mergesort$ from Sect.~\ref{sec:ad:sort:merge} is a typical example.
Here the $merge$ function is the aggregation of the results. 

A more difficult example is Strassen's multiplication algorithm from Sect.~\ref{sec:ad:matrix:strassen}.
It splits the arguments matrices into $4$ parts each.
Then it recurses $7$ times and aggregates the results.

\section{General Structure}

\subsection{Design}

A rather general class of divide-and-conquer algorithms is described as follows:
\begin{compactenum}
 \item input: problem of size $n$
 \item divide: create $r$ subproblems of size $n/d$
 \item conquer: solve the subproblems (usually recursively)
 \item return: combine the solutions of the subproblems into the solution of the overall problem and return it
\end{compactenum}

\subsection{Complexity}

%Let $C(n)$ be the time complexity for input of size $n$.
%Let $div(n)$ and $combine(n)$ be the costs of dividing and combining, and let $f(n)=div(n)+combine(n)$.
%Then we have
% \[C(n)=div(n) + r\cdot C(n/d) + combine(n)=r\cdot C(n/d) + f(n)\]
%
%By recursively substituting this formula into itself, we obtain
% \[C(n)=r\cdot C(n/d) + f(n) = r(r\cdot C(n/d^2) + f(n/d)) + f(n)\]
%\[= \ldots = r^k\cdot C(n/d^k) + r^{k-1}\cdot f(n/d^{k-1}) + \ldots + r\cdot f(n/b) + f(n)\]
%where $k=roundup(\log_d n)$, i.e., $n/d^k\leq 1$.
%Because the base cases $C(1)$ and $C(0)$ can always be solved in $O(1)$, we have $C(n/d^k)\in O(1)$.
%
%Such equations are called recurrences.
%Solving them means giving a closed from for $C(n)$.
%This is possible for many functions $f$, but the closed forms can become very complicated.
%
%If we are only interested in the $\Theta$-class of $C$ and $f$, the Master theorem handles most cases:
%Let $f\in \Theta(n^c)$ for some $c>0$, e.g., $f$ could be a polynomial of degree $c$.
%Then the recurrence becomes
% \[C(n)\in r^k\cdot O(1) + r^{k-1}\cdot \Theta(n^c/d^{c(k-1)}) + \ldots + r\cdot n^c/d^c + n^c\]
%We can now distinguish three cases:
%
%r^k=r^{\log_d n}=

\section{Examples}

\subsection{Binary Search}

Binary search checks whether a sorted list of length $n$ contains a certain value.
This takes $\Theta(\log n)$ if divide-and-conquer is used.

\subsection{Karatsuba's Multiplication}

We want to multiply two polynomials $p(x)=p_dX^d+\ldots+p_1X+p_0$ and $q(x)=q_dX^d+\ldots+q_1X+q_0$ with integer coefficients $p_i,q_i$.

Without loss of generality, we assume that $d=2^n-1$.
(If $d$ is not of this form, we can simply add $0$-terms to increase $d$.)
Thus, each polynomial can be seen as a list $[p_d,\ldots,p_0]$ of length $d+1=2^n$.

From now on, we represent polynomials as lists in this way.

\subsubsection{Base Cases}

If $n=0$, the lists have length $1$, i.e., the polynomials are constant integers.
Here we use normal integer multiplication: $pq=[p_0 q_0]$.

If $n=1$, we have $pq=[p_1 q_1, p_1 q_0 + p_0 q_1, p_0 q_0]$.
The naive computation takes $4$ multiplications and $1$ addition.
But we can compute it using only $3$ multiplications and $4$ additions as follows:
\[pq=[a, b-a-c, c] \tb\mwhere a=p_1q_1, \;b=(p_1+p_0)(q_1+q_0),\;c=p_0 q_0\]

\subsubsection{Algorithm}

Karatsuba designed the following divide-and-conquer algorithm:
\begin{compactenum}
 \item Let $k=(d+1)/2=2^{n-1}$.
 \item Split $p$ into two lists $p^u=[p_d,\ldots,p_k]$ and $p^l=[p_{k-1},\ldots,p_0]$ of length $k$. Split $q$ into $q^u$ and $q^l$ accordingly.\\
 Now (as polynomials) $p(X)=p^u X^k+p^l$ and $q(X)=q^u X^k+q^l$.
 \item Using the same trick as in the base case, compute $pq$ using $3$ multiplications of polynomials of lists of length $k$.
\end{compactenum}

% variant: n-bit int multiplication

\subsection{Associative Folding}\label{sec:ad:monoidfold:divide}

Consider folding over a monoid from Sect.~\ref{sec:ad:monoidfold}.

We can give a divide-and-conquer algorithm for it:

\begin{acode}
\afun[A]{monoidFold[A]}{mon: Monoid[A], x: List[A]}{
 \aifelse{empty(x)}{mon.e}{
   n := length(x)\\
   i := n \divop 2\\
   lower := [x_0,\ldots,x_{i-1}]\\
   upper := [x_i,\ldots,x_{n-1}]\\
   lowerFold := monoidFold(mon,lower) \\
   upperFold := monoidFold(mon,upper) \\
   mon.op(lowerFold, upperFold)
 }
}
\end{acode}

\paragraph{Correctness}
The correctness is straightforward.
The key insight is that associativity allows us to bracket the monoid operations any way we want, e.g.,
\[x_0 \;mon.op\;\ldots \;mon.op\; x_{i-1} \;mon.op\; x_i \;mon.op\; \ldots mon.op\; x_{n-1} =\]
\[(x_0 \;mon.op\; \ldots \;mon.op\; x_{i-1}) \;mon.op\; (x_i \;mon.op\; \ldots mon.op\; x_{n-1}) \]

\paragraph{Complexity}
Let us assume that we use arrays to split the lists in constant time.
Then the recurrence for the complexity is
 \[C(n)=2\cdots C(n/2)+O(1) \tb\mand\tb C(0)=C(1)= O(1)\]
Working out the recursion yields $\Theta(n)$, which is the same as in the naive case.

Thus, not every divide-and-conquer algorithm yields an improvement.

This is not surprising because all we do is change the bracketing.
The number of occurrences of $mon.op$ remains the same, i.e., we have to compute the monoid operation the same number of times.