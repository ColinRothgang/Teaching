\section{Overview}

A divide-and-conquer algorithm consists of $3$ steps:
\begin{compactenum}
 \item the input is divided into parts,
 \item each part is processed recursively,
 \item the results of the parts are aggregate into the result of the whole.
\end{compactenum}
Divide-and-conquer works whenever a problem of size $n$ can be reduced to multiple smaller subproblems of the same kind.

The simplest special case arises when working on a list that is divided into two parts.
$mergesort$ from Sect.~\ref{sec:ad:sort:merge} is a typical example.
Here the $merge$ function is the aggregation of the results. 

A more difficult example is Strassen's multiplication algorithm from Sect.~\ref{sec:ad:matrix:strassen}.
It splits the arguments matrices into $4$ parts each.
Then it recurses $7$ times and aggregates the results.

\section{General Structure}

\subsection{Design}

A rather general class of divide-and-conquer algorithms is described as follows:
\begin{compactenum}
 \item create $k$ subproblems of size $n/d$ in $D(n)$ time,
 \item 
\end{compactenum}

% example: Karatsuba Multiplication
% Karatsuba: multiply polynomials by using upper/lower coefficients (prelim: multiply linear polynomials with 3 multiplications), variant: n-bit int multiplication

\subsection{Complexity}
% Master theorem

\section{Examples}

\subsection{Binary Search}

Binary search checks whether a sorted list of length $n$ contains a certain value.
This takes $\Theta(\log n)$ if divide-and-conquer is used.

\subsection{Associative Folding}\label{sec:ad:monoidfold:divide}

Consider folding over a monoid from Sect.~\ref{sec:ad:monoidfold}.

We can give a divide-and-conquer algorithm for it:

\begin{acode}
\afun[A]{monoidFold[A]}{mon: Monoid[A], x: List[A]}{
 \aifelse{empty(x)}{mon.e}{
   n := length(x)\\
   i := n \divop 2\\
   lower := [x_0,\ldots,x_{i-1}]\\
   upper := [x_i,\ldots,x_{n-1}]\\
   lowerFold := monoidFold(mon,lower) \\
   upperFold := monoidFold(mon,upper) \\
   mon.op(lowerFold, upperFold)
 }
}
\end{acode}

\paragraph{Correctness}
The correctness is straightforward.
The key insight is that associativity allows us to bracket the monoid operations any way we want, e.g.,
\[x_0 \;mon.op\;\ldots \;mon.op\; x_{i-1} \;mon.op\; x_i \;mon.op\; \ldots mon.op\; x_{n-1} =\]
\[(x_0 \;mon.op\; \ldots \;mon.op\; x_{i-1}) \;mon.op\; (x_i \;mon.op\; \ldots mon.op\; x_{n-1}) \]

\paragraph{Complexity}
Let us assume that we use arrays to split the lists in constant time.
Then the recurrence for the complexity is
 \[C(n)=2\cdots C(n/2)+O(1) \tb\mand\tb C(0)=C(1)= O(1)\]
Working out the recursion yields $\Theta(n)$, which is the same as in the naive case.

Thus, not every divide-and-conquer algorithm yields an improvement.

This is not surprising because all we do is change the bracketing.
The number of occurrences of $mon.op$ remains the same, i.e., we have to compute the monoid operation the same number of times.