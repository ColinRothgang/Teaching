\documentclass[a4paper]{article}

\usepackage[course={Algorithms and Data Structures},number=10,date=2017-04-04,duedate=2017-04-13,duetime=11:00,unpublished]{../../myhomeworks}

\newcounter{chapter} % needed for dependencies of mylecturenotes
\usepackage[root=../..]{../../mylecturenotes}
\usepackage{../../macros/algorithm}

\begin{document}

\header

\begin{problem}{Records}{8}
Represent the record type and ... the record value ... in the following languages
\begin{compactitem}
 \item using structs in C
 \item using records in SML
 \item using class and instances in some object-oriented languages
 \item using dictionaries in Python
 \item using JSON objects in Javascript
\end{compactitem}

Naturally, untyped languages cannot handle record \emph{types}. So you can skip the type for Javascript and Python.
\end{problem}

\begin{problem}{Monoids}{8}
Implement the data structure of monoids over a type $A$., e.g., as a class

Implement the concrete instance $Matrix22Multtiplication$ (as given in the lecture).

Write a test program that computes the matrix $(F\cdot F)\cdot F$ by applying $(\anew{Matrix22Multtiplication}{}).op$ twice.
(Note that $F$ is the matrix for Fibonacci numbers.)
\end{problem}

\begin{problem}{Square-and-Multiply}{8}
Implement the square-and-multiply algorithm to compute powers in an arbitrary monoid (as specified in the lecture notes).

Test your function by computing large Fibonacci numbers logarithmically using
\[(1,0)\cdot sqmult(Matrix22Multtiplication, F, n)\]
where $F$ is as in the previous problem and $n$ is large.\footnote{You will probably want to use arbitrary-precision arithmetic for the integer operations to truly appreciate how efficient this algorithm is.}
\end{problem}

\end{document}
