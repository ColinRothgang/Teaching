\documentclass[a4paper]{article}

\usepackage[course={Algorithms and Data Structures},number=12,date=2017-05-09,duedate=2017-05-18,duetime=11:00]{../../myhomeworks}

\newcounter{chapter} % needed for dependencies of mylecturenotes
\usepackage[root=../..]{../../mylecturenotes}
\usepackage{../../macros/algorithm}

\begin{document}

\header

\begin{problem}{Greedy Algorithms}{12}
Using the structure of the generic greedy algorithm, implement the greedy algorithm for problem from Ex.~21.4.
%
%For example, you could use
%\begin{acode}
%\aclassA{WeightedMatroid}{size:\Int}{}{
%\afunI[\Int]{weight}{x:\Int}{}\\
%\afunI[\Bool]{acceptable}{x:Set[\Int]}{}
%}\\
%\\
%\afun[{Set[\Int]}]{greedy}{mat: WeightedMatroid}{
%  elements := \text{sort $[1,\ldots,mat.size]$ by $mat.weight$}\\
%  \ldots
%}
%\end{acode}
%where we assume that the set $M$ is given by $\{1,\ldots,mat.size\}$.
%
%Implement the matroids used for the scheduling algorithm (21.4.3).
%For example, you could use
%\begin{acode}
%\aclass{Task}{deadline:\Int, penalty: \Int}{}{}\\
%\aclassA{SchedulingMatroid}{n:\Int, tasks: List[Task]}{Matroid(n)}{
%\afunI[\Int]{weight}{x:\Int}{tasks[x].penalty}\\
%\afunI[\Bool]{acceptable}{x:Set[\Int]}{
%  \text{use criterion from Thm.~21.4}
%}
%}
%\end{acode}
\end{problem}

\begin{problem}{Matroids}{6}
Prove that the structure used in the scheduling algorithm (21.4.3) is indeed a matroid.
\end{problem}


% homework: coin change problem, greedy vs. dynamic

\begin{problem}{Dynamic Programming}{}
TBD
\end{problem}

\begin{problem}{Parallelization}{6}
Consider the parallel associative folding algorithm from the notes.

Determine the time complexity $C^k(n)$ for runing it on a list of length $n$ using $k$ machines.
\end{problem}

\end{document}
