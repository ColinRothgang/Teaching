Quantum algorithms use a different physical machine, namely a quantum computer that exploits quantum effects.
In the most important idea, the superposition of quantum states is used: that allows performing a computation on multiple inputs at the same time, from which a single result value is obtained through measurement.

Quantum algorithms do not allow computing more, i.e., the definition of computable and decidable is not affected.
But they can solve certain problems faster, e.g., we can give polynomial quantum algorithms for some classically super-polynomial problems.
That makes them interesting for many application, in particular in cryptography where the non-existence of polynomial solutions is often used as a criterion for security.

For example, a quantum algorithm can be used to test $f(0)=f(1)$ in such a way that $f$ is executed only once.
We call $f$ on the superposition of the $0$ and $1$ states of a quantum bit to compute the superposition of $f(0)$ and $f(1)$.
Then we perform a certain quantum transformation that maps superpositions of equal values t0 $0$ and superpositions of different values to $1$.
Eventually, we measure the result.

Quantum algorithms were hypothetical until researchers succeeded in building quantum computers in recent years.
Existing quantum computers are still small (e.g., $16$ bits), and it remains open if and when they can be used cost-effectively for large problems.