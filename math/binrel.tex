A binary relation on a set $A$ is a subset $\#\sq A\times A$.
We usually write $(x,y)\in\#$ as $x\# y$.

\subsection{Classification}

\begin{definition}[Properties of Binary Relations]\label{def:math:binrel}
We say that $\#$ is \ldots if the following holds:
 \begin{compactitem}
 \item reflexive:  for all $x$, $x\# x$
 \item irreflexive:  for no $x$, $x\# x$
 \item transitive: for all $x,y,z$, if $x\# y$ and $y\# z$, then $x\# z$
 \item a strict order: irreflexive and transitive
 \item a preorder: reflexive and transitive
 \item anti-symmetric: for all $x,y$, if $x\# y$ and $y\# x$, then $x=y$
 \item symmetric: for all $x,y$, if $x\# y$, then $y\# x$
 \item an order\footnotemark: preorder and anti-symmetric
 \item an equivalence: preorder and symmetric
 \item a total order: order and for all $x,y$, $x\# y$ or $y\# x$
 \end{compactitem}

An element $a\in A$ is called \ldots of $\#$ if the following holds:
 \begin{compactitem}
  \item least element:  for all $x$, $a\# x$
  \item greatest element: for all $x$, $x\# a$
  \item least upper bound for $x,y$: $x\# a$ and $y\# a$ and for all $z$, if $x\# z$ and $y\# z$, then $a\# z$
  \item greatest lower bound for $x,y$: $a\# x$ and $a\# y$ and for all $z$, if $z\# x$ and $z\# y$, then $z\# a$
 \end{compactitem}
\end{definition}
\footnotetext{Orders are also called \emph{partial order}, \emph{poset} (for partially ordered set), or \emph{ordering}.}

\begin{definition}[Dual Relation]
For every relation $\#$, the relation $\#^{-1}$ is defined by $x\,\#^{-1}\, y$ iff $y\,\#\,x$.

$\#^{-1}$ is called the \textbf{dual} of $\#$.
\end{definition}

\begin{theorem}[Dual Relation]
If a relation is reflexive/irreflexive/transitive/symmetric/antisymmetric/total, then so is its dual.
\end{theorem}

\subsection{Equivalence Relations}

Equivalence relations are usually written using infix symbols whose shape is reminiscent of horizontal lines, such as $=$, $\sim$, or $\Equiv$.
Often vertically symmetric symbols are used to emphasize the symmetry property.

\begin{definition}[Quotient]
Consider a relation $\Equiv$ on $A$.
Then
\begin{compactitem}
 \item For $x\in A$, the set $\{y\in A\,\|\,x\Equiv y\}$ is called the (equivalence) \textbf{class} of $x$.
  It is often written as $[x]_\Equiv$.
 \item $A/\Equiv$ is the set of all classes.
  It is called the \textbf{quotient} of $A$ by $\Equiv$.
\end{compactitem}
\end{definition}

\begin{theorem}
For a relation $\Equiv$ on $A$, the following are equivalent\footnote{Logical equivalence is itself an equivalence relation.}:
\begin{compactitem}
 \item $\Equiv$ is an equivalence.
 \item There is a set $B$ and a function $f:A\to B$ such that $x\Equiv y$ iff $f(x)=f(y)$.
 \item Every element of $A$ is in exactly one class in $A/\Equiv$.
\end{compactitem}

In particular, the elements of $A/\Equiv$ 
\begin{compactitem}
 \item are pairwise disjoint,
 \item have $A$ as their overall union.
\end{compactitem}
\end{theorem}

\subsection{Orders}

\begin{theorem}[Strict Order vs. Order]
For every strict order $<$ on $A$, the relation ``$x<y$ or $x=y$'' is an order.

For every order $\leq$ on $A$, the relation ``$x\leq y$ and $x\neq y$'' is a strict order.
\end{theorem}

Thus, strict orders and orders come in pairs that carry the same information.

Strict orders are usually written using infix symbols whose shape is reminiscent of a semi-circle that is open to the right, such as $<$, $\subset$, or $\prec$.
This emphasizes the anti-symmetry ($x< y$ is very different from $y<x$.) and the transitivity ($< \ldots <$ is still $<$.)
The corresponding order is written with an additional horizontal bar at the bottom, i.e., $\leq$, $\sq$, or $\preceq$.
In both cases, the mirrored symbol is used for the dual relation, i.e., $>$, $\supset$, or $\succ$, and $\geq$, $\supseteq$, and $\succeq$. 

\begin{theorem}\label{thm:math:binrel}
If $\leq$ is an order, then least element, greatest element, least upper bound of $x,y$, and greatest lower bound of $x,y$ are unique whenever they exist.
\end{theorem}

\begin{theorem}[Preorder vs. Order]
For every preorder $\leq$ on $A$, the relation ``$x\leq y$ and $y\leq x$'' is an equivalence.

For equivalence classes $X$ and $Y$ of the resulting quotient, $x\leq y$ holds for either all pairs or no pairs $(x,y)\in X\times Y$.
If it holds for all pairs, we write $X\leq Y$.

The relation $\leq$ on the quotient is an order.
\end{theorem}

\begin{remark}[Totality]
If $\leq$ is a preorder, then for all elements $x,y$, there are four mutually exclusive options:

\begin{ctabular}{|l|c|c|c|}
\hline
& $x\leq y$ & $x\geq y$ & $x=y$ \\
\hline
$x$ strictly smaller than $y$, i.e., $x>y$ & true  & false & false\\
$x$ strictly greater than $y$, i.e., $x<y$ & false & true  & false \\
$x$ and $y$ incomparable      & false & false & false \\
$x$ and $y$ similar           & true  & true  & maybe \\
\hline
\end{ctabular}
Now anti-symmetry excludes the option of similarity (except when $x=y$ in which case trivially $x\leq y$ and $x\geq y$).
And totality excludes the option of incomparability.

Combining the two exclusions, a total order only allows for $x>y$, $y<x$, and $x=y$.
\end{remark}


