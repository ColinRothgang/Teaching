\section{Overview}

Two sub-fields can be distinguished.

\paragraph{Verification Environments}
These tools are specifically designed to prove correctness of programs.

The programming language may be designed for the tool, or the tool may be designed to work with an existing programming language.
The used programming languages tend to be simpler and more cleanly designed than mainstream languages.
Often they arise as well-behaved fragments of existing languages.

They tend to focus on the programming aspects, allowing for user-facing syntax that looks as similar as possible to typical programs.
They tend to try to hide the logic aspects from the user as much as possible.
Ideally, they only require the user to ascribe loop invariants, class invariants, and pre/postconditions for functions.

Then automated provers can perform the proving in the background.
Usually no proof object is produced.

Examples are Why3 (for a simple imperative language) and Key (for Java).

\paragraph{Proof Assistants}
These are general-purpose tools for proving formulas.
Many use formal systems that are strong enough to be Turing-complete.

They tend to focus on \emph{pure} programs, i.e., programs where evaluation does not require an environment.
These are usually functional programming languages emphasizing inductive data types and recursive functions.
Assignments and I/O are avoided.

The correctness of recursive functions is usually shown using induction arguments.
Non-termination is the primary source of non-correctness.
To prove termination, a variety of criteria have been developed, mostly based on certain well-orderings.

All proofs are interactive, and the proof assistant supplies as much automation as possible.
The result is a proof term that can be verified independently.

Examples are Coq, Agda, PVS, and HOL family, which includes Isabelle, HOL Light, and HOL.