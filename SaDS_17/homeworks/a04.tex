\documentclass[a4paper]{article}

\usepackage[course={Secure and Dependable Systems},number=4,date=2017-03-15,duedate=2017-03-23,unpublished]{../../myhomeworks}

\newcounter{chapter} % needed for dependencies of mylecturenotes
\usepackage[root=../..]{../../mylecturenotes}
\usepackage{../../macros/algorithm}

\begin{document}

\header

\begin{problem}{Fixing C}{3}
Try to give productions, typing rules, and evaluation rules for C-style pointers.

This is not entirely possible because C is mis-designed from a formal system perspective.

But it can be done partially:
\begin{compactitem}
 \item for every type $A$, there should be a type $*A$ of pointers to locations holding objects of type $A$,
 \item for every pointer $p:*A$, there should be a type $p*:A$ that evaluates to the value at location $p$,
 \item for every (this is tricky/impossible depending on how you try it) name $x:A$, there should be a pointer $\&x:*A$ such that $(\&x)*==x$,
 \item for every pointer $p:*A$ and integer $n$, there should be a pointer $p+n$ (this is impossible in general but may work for special cases of $A$ and $n$)
\end{compactitem}

Any partial solution yields a well-behaved fragment of $C$.
\end{problem}

\begin{problem}{Dynamic Logic: Practice}{3}
Install either Why3 or KeY (see the links in the lecture notes).

Write a simple program that contains a while-loop and use the tool to verify that it meets its specification.
\medskip

Submit a reasonable combination of screen shots, shell logs, system output etc. that demonstrates you completed the task.

You may use any example that is already part of the available documentation or tutorials.
But you have to prove that you actually installed the system and ran the verification.
For example, you can copy an example from the tutorial, rename the function to your name, and then run the verification.
\end{problem}


\begin{problem}{Dynamic Logic: Theory}{4}

\end{problem}

\end{document}
