\message{ !name(a03.tex)}\documentclass[a4paper]{article}

\usepackage[course={Secure and Dependable Systems},number=3,date=2017-02-28,duedate=2017-03-14]{../../myhomeworks}

\newcounter{chapter} % needed for dependencies of mylecturenotes
\usepackage[root=../..]{../../mylecturenotes}
\usepackage{../../macros/algorithm}

\begin{document}

\message{ !name(a03.tex) !offset(-3) }


\header

\begin{problem}{Formal Systems: Practice}{3}
Consider the partial implementation of a formal system in the course repository.
It already includes most algorithms and features discussed in the notes.

Extend this implementation with product types.
That requires changing the data types, parser, printer, checker, and interpreter.
\end{problem}

\begin{problem}{Formal Systems: Theory}{4}
Consider the type theory and programming language developed in the lecture notes.
\begin{enumerate}
\item Give two meaningful programs as terms over the programming language developed in the lecture notes.

The details of the programs are not essential but one program each should consist of (at least):
\begin{compactitem}
 \item a recursive function declaration,
 \item a function declaration that contains a while-loop that makes assignments to a mutable variable
\end{compactitem}
For example, you can implement the factorial function once recursively and once using a while-loop.

\item Give the derivations for the well-formedness of your programs in all detail.
\end{enumerate}

Note: The typing derivations are trees that tend to grow quickly and can be painful to write down.
For example, you may:
\begin{compactitem}
  \item write them manually on paper and submit a photograph or scan, or
  \item write them in LaTeX, or
  \item adapt the Checker.scala class to print out the derivation in some format (svg, LaTeX, \ldots).
\end{compactitem}
\end{problem}

\begin{problem}{Formal Systems: Theory}{3}
Add disjoint-union types to the type theory by giving new
\begin{compactitem}
 \item productions to the grammar,
 \item typing rules,
 \item evaluation rules.
\end{compactitem}

The disjoint-union type 
\message{ !name(a03.tex) !offset(-1) }

\end{document}
