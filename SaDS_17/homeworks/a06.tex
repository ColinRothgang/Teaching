\documentclass[a4paper]{article}

\usepackage[course={Secure and Dependable Systems},number=6,date=2017-04-27,duedate=2017-05-04]{../../myhomeworks}

\newcounter{chapter} % needed for dependencies of mylecturenotes
\usepackage[root=../..]{../../mylecturenotes}
\usepackage{../../macros/algorithm}

\begin{document}

\header

\begin{problem}{Practice: Building an Encryption Scheme}{5}
Implement abstract classes for
\begin{compactitem}
 \item symmetric encryption schemes
 \item block ciphers
\end{compactitem}

Implement concrete classes for
\begin{compactitem}
 \item the block cipher from the exercise in Sect.~15.3.3 in the lecture notes
 \item the encryption scheme that takes a block cipher and the IV and uses the CBC mode of operation
\end{compactitem}

Combine the two concrete classes into an encryption scheme and write a unit test that checks the inversion condition: randomly generate a sequence of blocks, encrypt it, decrypt the ciphertext, and check for equality.
\end{problem}

\begin{problem}{Practice: Relevance of Modes of Operation}{2}
Use your implementation from the previous problem to encrypt a file.

This should be a real file in an uncompressed format, e.g., a bitmap image.
It should be big enough to consist of many blocks.

Modify your implementation to use the trivial mode of operation (where no IV is used and each block is simply passed to the block cipher).
Encrypt the same file with this mode as well and compare both results with the original.

Note: This homeworks aims at reproducing the effect from the penguin image example at \url{https://en.wikipedia.org/wiki/Block_cipher_mode_of_operation#Electronic_Codebook_.28ECB.29}.
\end{problem}

\begin{problem}{Theory: Security Analysis}{3}
Consider the block cipher from the exercise in Sect.~15.3.3 but using only $1$ round.

We define an encryption scheme using the trivial mode of operation that chooses an $8$-bit key $k$ and then encrypts every $8$-bit block by applying the block cipher.

Informally prove the following
\begin{enumerate}
  \item $E$ is comp-ind secure if $k$ is chosen with a PRG.\\
    Hint: This is already true for networks that use no substitution or permutation steps and only one key step.
  \item $E$ is \emph{not} CPA-ind secure.\\
   Hint: The adversary can perform a chosen-plaintext attack to recover the key by encrypting a self-chosen message.
\end{enumerate}
\end{problem}

\end{document}
