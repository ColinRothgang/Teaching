\documentclass[a4paper]{article}

\usepackage[course={Secure and Dependable Systems},number=6,date=2017-04-27,duedate=2017-05-04,unpublished]{../../myhomeworks}

\newcounter{chapter} % needed for dependencies of mylecturenotes
\usepackage[root=../..]{../../mylecturenotes}
\usepackage{../../macros/algorithm}

\begin{document}

\header

\begin{problem}{Practice: Building an Encryption Scheme}{5}
Implement abstract classes for
\begin{compactitem}
 \item symmetric encryption schemes
 \item block ciphers
\end{compactitem}

Implement concrete classes for
\begin{compactitem}
 \item the block cipher from the example in the lecture
 \item the encryption scheme that takes block cipher and uses the CBC mode of operation
\end{compactitem} 

Write a unit test that checks the inversion condition: randomly generates some blocks, encrypt and decrypt them, and check for equality.
\end{problem}

\begin{problem}{Practice: Relevance of Modes of Operation}{2}
Use your implementation from the previous problem to encrypt a file.

This should be a real file in an uncompressed format, e.g., a bitmap image.
It should be big enough to consist of many blocks.

Modify your implementation to use the trivial mode of operation (where no IV is used and each block is simply passed to the block cipher).
Encrypt the same file with this mode and compare both results with the original.

Note: This homeworks aims at reproducing the effect from the penguin image example at \url{https://en.wikipedia.org/wiki/Block_cipher_mode_of_operation#Electronic_Codebook_.28ECB.29}.
\end{problem}

\begin{problem}{Theory}{3}

\end{problem}


\end{document}
