The basic principle of symmetric encryption is that encryption and decryption use the same key.

\subsection{Schemes}

We now expand on the ideas developed in Sect.~\ref{sec:sd:crypto:hist} systematically.

\begin{definition}[Encryption Scheme]
 An \textbf{encryption scheme} is a tuple $(\Sigma,K,G, E, D)$, where
  \begin{compactitem}
   \item $\Sigma$ is a set (the \textbf{alphabet}),
   \item $K=(K_n)_{n\in \N}$ is a family of sets (the \textbf{key space}),
   \item $G:(n\in \N)\to K_n$ is a PPT algorithm (the \textbf{key generation} function)
   \item $E=(E_k)_{n\in\N,k\in K_n}$ is a family of polynomial algorithms $E_k:\Sigma^n\to\Sigma^*$ (the \textbf{encryption} functions)
   \item $D=(D_k)_{n\in\N,k\in K_n}$ is a family of (possibly partial) polynomial algorithms $D_k:\Sigma^*\to\Sigma^n$ (the \textbf{decryption} functions)
  \end{compactitem}
  such that for all $n\in N$, $k\in K_n$, and $x\in \Sigma^n$, we have $D_k(E_k(x))=x$.

  For $x\in\Sigma^n$, we write $E(x)$ for the probabilistic result of computing $E_{G(n)}(x)$.
\end{definition}

To encrypt a message $x$ of length $n$, we choose a key $G(n)\in K_n$ and call $c=E_k(x)$ to obtain the cipher $c$.
To decrypt an encrypted message, we call $D_k(c)$.

\subsection{Security of a Scheme}

There are various concepts of security of an encryption scheme.
The general idea is to assume an adversary that picks two messages $x_0,x_1\in\Sigma^n$ and randomly receives either $E(x)$ or $E(x')$.
The encryption scheme is consider secure if the adversary cannot distinguish between the two situations with a probability that is non-negligibly better than $1/2$.
In other words, even if the adversary already knows that a given ciphertext $c$ is either the encryption of $x$ or of $x'$, he has no better chance of decrypting $c$ than guessing.

In all cases, the adversary is restricted to polynomial computations.
But we obtain different notions of security depending on how we model what else the adversary is allowed to do.

In the simplest case, the adversary may do nothing else:

\begin{definition}[Guess-indistinguishable]\label{def:sd:ind}
  An encryption scheme $(\Sigma,K,G,E,D)$ is \textbf{guess-indistinguishable} if for any PPT $A:\Sigma^*\to\{0,1\}$, messages $x_0,x_1\in\Sigma^n$, and $n\in \N$
  \[\Prob{i\in\{0,1\}}{A(E(x_i))=i}<\frac{1}{2}+neg(n)\]
  for a negligible function $neg$.
\end{definition}

Here $\Prob{i\in\{0,1\}}{A(E(x_i))=i}$ formalizes the probability that the adversary correctly guessed whether $x_0$ or $x_1$ is the decryption of its input.

\begin{example}
A substitution cipher is not secure in the sense of Def.~\ref{def:sd:ind}.
We define an adversary $A$.
Given the encryption $c$ of $x_0$ or $x_1$, $A$ collects the frequencies of all characters in $c$.
If those frequencies match the frequencies in $x_0$ or $x_1$, $A$ guesses $0$ or $1$, respectively.
Clearly, $A$ is a polynomial algorithm.
\end{example}

Guess-indistinguishability is still a relatively weak notion of security because a realistic adversary may have access to the encryption scheme and may try to reverse-engineer it in some way.
If the adversary has access to the encryption function $E(-)$, we speak of a \textbf{chosen-plaintext-attack} (CPA).
The analog of Def.~\ref{def:sd:ind} where the adversary $A$ may conduct CPAs is called CPA-ind.
If the adversary additionally has access to the decryption function $D(-)$, we speak of \textbf{chosen-ciphertext-attack} (CCA).
The analog of Def.~\ref{def:sd:ind} where the adversary $A$ may conduct CCAs as well is called CCA-ind.

\begin{example}
Let $\Sigma=\{0,1\}$.
We consider every natural number to be an element of $\{0,1\}^*$ by using its binary representation.

Given a PRG $R$, we can iterate it on its own output to get an arbitrarily long pseudo-random sequence in $\{0,1\}^*$.
Now we can construct an encryption scheme by simply defining the key to be $G(n)=R(n)$ and defining $E_k(x)$ by xoring every bit in $x$ with the corresponding bit in $k$.
The decryption can be done in the same way, i.e., $D_k=E_k$.

The resulting encryption scheme is computationally indistinguishable but not necessarily CPA-ind.
\end{example}

\subsection{Schemes Based on Block Ciphers}

We fix the alphabet to be $B=\{0,1\}$.

\subsubsection{Block Ciphers and Modes of Operation}

Block ciphers are a common method to obtain more secure encryption schemes.

A \textbf{block cipher} is a function that maps keys to bijections of the set $B^N$ for some $N$.
The elements of $B^N$ are called \textbf{blocks}.

The idea of block cipher--based schemes is to split the plaintext into blocks that are translated individually by applying the block cipher.
The last block may have to be \textbf{padded} to length $N$ by adding random data.

However, the naive approach would not yield secure schemes---if the same block is always encrypted in the same way, the scheme would be easy to attack.
Overcoming this is the role of the block cipher \textbf{mode of operation}.
There are various modes that yield CPA-ind secure schemes if used with pseudo-random block ciphers.

A commonly used mode of operation is \textbf{Cipher Block Chaining} (CBC).
Here every plaintext block is xor-ed with a certain element from $B^N$ before applying the block cipher.
For the first plaintext block, this is an arbitrary number called the \textbf{initialization vector}.
For every subsequent plaintext block, it is the previous cipher block.

The initialization vector must be random but does not have to be secret (which is good because the recipient needs to know it to decrypt).
To maintain security, the same pair of initialization and key must never be used twice, i.e., the initialization vector should be a \textbf{nonce} (a number only used once).
For example, it could be derived from the number of the current message in the overall sequence of exchanged messages.

%\subsubsection{Feistel Ciphers}
%
%For this we can use a so called Feistel network.
%%Improving a comp. ind. encryption scheme to an ind. CPA secure scheme using a Feistel-network.
%\begin{definition}[A Feistel cipher]
% Let $k$ be any natural number (the number of \emph{rounds}). Let $f_{k_i}$ be a family of functions\protect\footnote{If possible pseudorandom functions and possibly one-way functions.} ($f$ is the so called \emph{round function}) of output length $n$ indexed by the sequence of \emph{round keys} $k_1, k_2, \ldots, k_n$. 
% Then the following encryption algorithm $E_k$ is called Feistel cipher. %network with $k$ iterations (for some odd $k$) based on a pseudo random generator $f_{k_i}$ and round keys $k_1, k_2, \ldots, k_n$ is an encryption scheme $(G,E,D)$, defined by:
% %For any odd $k\in\mathbb{N}$ we call an encryption scheme $(G, E, D)$ a Feistel network with $k$ iterations and $P$-Box $f$, iff for some \emph{round keys} $k_1, k_2, \ldots, k_n$ used as input for the pseudo random generator $f_{k_i}$. 
% \begin{itemize}
%  \item Fix a message $m=:x_1\circ x_2$, where $\left|x_1\right|=\left|x_2\right|=n$
%  \item Define the sequences $L_1, L_2, \ldots, L_n$ and $R_1, R_2, \ldots, R_n$ by $L_1:=x_1, R_1:=x_2$ and $L_{n+1}:=R_n, R_{n+1}:=L_n\oplus f_{k_n}(R_n)$. Finally define $E_k:x_1\circ x_2\to L_k\circ R_k$. 
% \end{itemize}
% Now we can define the corresponding decryption algorithm $D_k$ just like $e_k$, but with the reversed order of round keys:
% \begin{itemize}
%   \item Fix a ciphertext $c=:x_1\circ x_2$, where $\left|x_1\right|=\left|x_2\right|=n$
%   \item Define the sequences $L_1, L_2, \ldots, L_n$ and $R_1, R_2, \ldots, R_n$ by $L_1:=x_1, R_1:=x_2$ and $L_{n+1}:=R_n, R_{n+1}:=L_n\oplus f_{k_{k-n}}(R_n)$. Finally define $D_k:x_1\circ x_2\to L_k\circ R_k$. 
% \end{itemize}
%\end{definition}
%Feistel ciphers have been shown to fulfill several notions of security assuming that the round function is actually pseudo random. For instance Feistel networks with at least $3$ rounds are ind. CPA secure and for more rounds they fulfill even stronger notions of security. %  TODO: Check and clearify the exact model (3 seems sufficient under some assumptions, but 4 is definitely better (and already fulfills stronger notions)) and perhaps mention some other results. %see \url{https://link.springer.com/chapter/10.1007/978-3-540-45146-4\_30}

\subsubsection{Substitution-Permutation Networks}

A substitution-permutation network is a block cipher whose bijections arise as products of substitutions and permutation ciphers.

To process a block of $N$ bits, the block is divided into $b$ chunks of $n=N/b$ bits each.
Each block is processed by a sequence of steps, each of which applies a bijection $B^N\to B^N$.

A \textbf{substitution step} consists of \textbf{S-boxes} $S_1,\ldots,S_b$.
Each S-box is a bijection $B^n\to B^n$.
The substitution step maps each chunk by applying the corresponding S-box.
The S-boxes could be substitution ciphers.
However, it is desirable to have every output bit of an S-box depend on \emph{every} input bit.
Then changing one input bit maximally \textbf{confuses} the output.

A \textbf{permutation step} consists of one \textbf{P-box} $P$.
A P-box is a permutation of $\{0,\ldots,N-1\}$.
The permutation step maps the block by permuting its bits according to the P-box.
It is desirable that the bits of one chunk are rearranged to as many different chunks as possible.
That maximizes the \textbf{diffusion} of bits among the chunks.

A \textbf{key step} consists of one key $k\in B^N$.
The key steps maps a block by xor-ing it with $k$.

A \textbf{round} is a bijection of $B^N$ that is the product of a substitution step, a permutation step, and a key step.

A \textbf{network} is a sequence of rounds.
Often the substitution and permutation steps are the same for each round, and only the key step changes between rounds.
In that case, the keys are called the \textbf{round keys}.

The inverse of a network is defined by inverting all operations in reverse order.

%Substitution-permutation-network and Feistel networks using $S$-Boxes are quite similar.
%Ciphers based on substitution-permutation-networks can be better parallelized, but Feistel ciphers can use any pseudo-random function (for instance any one-way function) and are therefore limited to invertible ($P$-Boxes). %Also the Feistel networks can be adapted to ciphers not using blocks (for instance it is used in OAEP).

\subsubsection{AES}

\paragraph{Overview}
AES (Advanced Encryption Standard) was chosen by NIST (the US institute of standards and technology) in 2001 as an encryption standard after an open call in 1997 and extensive analysis of the submitted schemes.
Before being adopted as AES, it was called Rijndael.
It replaced DES, which was not secure anymore.

AES is one of the most widely used block ciphers, approved by many government organizations.
Implementations are available in many programming languages.

AES essentially uses a substitution-permutation network for $N=128$, $b=16$, and $n=8$.

\paragraph{Keys}
There exist three versions of AES for different key-sizes.
They differ mainly by the key size and the number of rounds:
\begin{center}
  \begin{tabular}{|c|c|c|c|}
  	\hline key size & 128-bit & 192-bit & 256-bit \\ 
  	\hline number of rounds & 10 & 12 & 14 \\ 
  	\hline
  \end{tabular}
\end{center}
In all three cases, one additional initial round is run that only xors the input with a round key.

Thus, $11$, $13$, or $15$ round keys of $128$ bits each are needed.
These are obtained from the overall key using the Rijndael key schedule, which we omit here.

\paragraph{Details}
In the following, let $n$ denote the number of rounds of the AES cipher.

The $128$-bit block of data to be encrypted is represented as a \emph{state} consisting of $16$ chunks of $8$ bits arranged as a $4\times 4$ matrix.

All rounds except for the initial and the final round are the product of four basic ciphers: sub-bytes, shift-row, mix-columns, and add-round-key.
The initial round skips the first three operations, and the final round skips the mix-columns operation.

The basic operations are defined as follows:
\begin{compactenum}
  \item Sub-bytes: This is a substitution step that applies a fixed S-Box (the Rijndael S-box, which we omit here) to every chunk of the state.
  \item Shift-row: This is a permutation step that leaves the chunks as they are but rearranges them relative to each other.
    Specifically, for $i=0,1,2,3$, the $i$-th row is left-shifted cyclically $i$ times.
  \item Mix-columns: This is a more complex substitution step.
    It applies the same fixed operation to each column of the state.
    
    This operation is defined as follows.
    A column consists of $4$ bytes, which can be seen as elements of $F_{2^8}$ (see Sect.~\ref{sec:math:finfield} for finite fields).
    Thus, a column can be seen as a $4$-dimensional vector over $F_{2^8}$.
    This vector is multiplied with the fixed $4\times 4$ matrix over $F_{2^8}$ given by
    \[\left(\begin{matrix}
    2&3&1&1\\1&2&3&1\\1&1&2&3\\3&1&1&2
   \end{matrix}\right)\]
  \item Add-round-key: The state is xored with the round-key.   
\end{compactenum}