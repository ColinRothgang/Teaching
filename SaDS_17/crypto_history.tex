\subsubsection{Substitution Ciphers}

One of the oldest known encryption algorithms is the so called Caesar cipher. It is said that he used it for communication with his army. It is a very simple character-wise substitution cipher. The idea is to substitute letters for each others.

\begin{example}[Caesar Cipher]
 In a very simple case the alphabet can be shifted by $3$ letters forward cyclically.
 Thus, \texttt{a} would be encrypted as \texttt{d}, \texttt{b} as \texttt{e}, \texttt{z} as \texttt{c}, etc.

 Below the lower row contains the encryption of the upper row. 
  \begin{lstlisting}
    meet me after the toga party
    phhw ph diwhu wkh wrjd sduwb
  \end{lstlisting}
\end{example}

Let us make this algorithm mathematically a bit more precise.
We usually represent all data as a number in $\Z_n$ so that encryption and decryption are functions $\Z_n\to\Z_n$.
$\Z_n$ is called the \textbf{alphabet}.
For the Latin alphabet, we have $n=26$, i.e., the letters are represented as the numbers $0, 2, 3,\ldots, 25$.

The length of the shift can be any number from $0$ to $25$.
It is very typical that an encryption scheme has such an arbitrary parameter.
Every choice of the parameter yields a different encryption function, and it is necessary to know that number to decrypt.
This number is called the \textbf{key}.

Now for a given key $k$, we can encrypt each letter $x$ as $\phi_k(x)=(x+k)\modop n$.
This is obviously, a very weak cipher.
Here, an attacker can easily try out all $26$ possible keys $k$ until the decryption of the message becomes a legible message.
Trying every possible key like this is called a \textbf{brute-force-attack}. 

We can generalize this approach of substituting the letters individually by different letters in the alphabet.
As the encryption function, we use an arbitrary bijection $\phi:\Sigma\to\Sigma$.
This is called a \textbf{substitution cipher}.
A \textbf{plaintext} or \textbf{message} is an element of $\Z_n^*$, i.e., a list of elements of the alphabet.
A plaintext is \textbf{encrypted} by applying $\phi$ to each element, which yields the \textbf{ciphertext}.

More precisely, this is called a \textbf{monoalphabetic} cipher.
There are $|\Sigma|$ different permutations, e.g., $\approx 4\times 10^{26}\approx 2^{88}$ for the Latin alphabet.
So a brute force attack now seems very hard and this looks like a very strong cipher, at first sight. 
However, monoalphabetic substitution ciphers can be attacked very efficiently because they do not obfuscate patterns in the plaintext---they encrypt the same plaintext with the same ciphertext every time.
In particular the same letter is always encrypted with the same letter. 
Thus, if the language of the message is known, the frequencies of occurrences of letters in the ciphertext can be correlated to the expected frequencies of occurrences of letters in the plaintext.
Then, if the ciphertext is long enough, only a few substitution remain likely and can be tried easily.

This is why \textbf{polyalphabetic} ciphers were introduced.
These encrypt entire blocks of the plaintext.
A very simple polyalphabetic cipher is the so called Vigin\`ere-cipher, a generalization of Caesar's cipher.
Here blocks of subsequent letters are encrypted using Ceasar's cipher using a different key for each letter in the block.
Thus, the key is a tuple $k=(k_0,\ldots,k_{l-1})$ and every character $x$ in position $i$ of the plaintext is encrypted as $\phi_k(i,x)=(x+k_{i\modop l})\modop n$.

\begin{example}[Vigin\`ere Cipher]
 Let use the key $k=(1,2,3)$.
 Then we obtain the encryption
  \begin{lstlisting}
    meet me after the toga party
    nghu oh bhwft wig wpid qcuua
  \end{lstlisting}
\end{example}
This can be attacked by first looking for repetitive patterns in the ciphertext (in order to guess the length $l$ of the key) and attacking the decomposed Caesar encrypted subsequences individually. 
This works well if the length of the message is significantly longer than the length of the key. 

One notable special case arises when the length of the key equals the length of the message.
In this case we call the cipher a \textbf{one-time-pad}.
Then the encryption is absolutely secure as every ciphertext can be decrypted to an arbitrary plaintext by choosing an appropriate key.
The security of course only holds if the key is only used once---otherwise, it is effectively to used to encrypt one very long message.
Therefore, one-time-pads have only limited use in practice: to send a second message, we have to transfer the new key secretly, which is as difficult as the original problem.
The only practical way to use a one-time-pad is to pre-agree on an extremely long key that is used up gradually as messages are sent.

\subsubsection{Transposition Ciphers}

A different approach is to \emph{rearrange} the letters of the message instead of substituting them.
That is called a \textbf{transposition} cipher.
An old example is the rail-fence transposition cipher.
Here a message is spelled out diagonally over a number of rows, and the ciphertext is read off row-by-row.
The key $k$ is the number of rows.

\begin{example}[Rail-Fence Cipher]
 If we use $k=2$, we obtain the following encryption
  \begin{lstlisting}
    meet me after the toga party
    me eatrtetg atetm fe h oapry
  \end{lstlisting}
which is computed using the following diagonal transposition
  \begin{lstlisting}
    m e   e a t r t e t g   a t
     e t m   f e   h   o a p r y
  \end{lstlisting}
\end{example}
Another interesting polyalphabetic cipher is the playfair cipher, which we omit here.

Formally, a transposition cipher encrypts a plaintext $[x_0,\ldots,x_{m-1}]$ of as $[x_{k(0)},\ldots,x_{k(l-1)}]$ where the key $k$ is a bijection of the positions in the message.

Transposition ciphers do not really obfuscate plaintext patterns either.
For instance, the number of occurrences of the letters remain unchanged by the cipher.

\subsubsection{Product Ciphers}
 
A good way to improve more secure encryptions is to compose multiple ciphers.
The resulting ciphers are called \textbf{product} ciphers.
If we compose a substitution cipher with another substitution cipher, we obtain a substitution cipher again, which is useless.
Similarly, if we compose two transposition ciphers, we get a transposition cipher again.

However, if we compose a substitution and a transposition, we get a new kind of cipher that is much harder to break.
This is a key idea in modern (symmetric) cryptography. 
