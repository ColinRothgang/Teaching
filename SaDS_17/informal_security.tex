\section{Social and Institutional Aspects}

There are roughly two ways to protect sensitive systems and data.

Firstly, the data can be stored in a way that it is only useful for the people meant to use it.
This may involve obfuscation or intentional non-documentation. Doing this systematically leads to cryptography.

Secondly, access to the data can be restricted.
This may involve physical restrictions of people (e.g., locks and ID checks) and logical restrictions of user accounts (e.g., read and write access rights to files or databases).

All aspects are interrelated.
For example:
\begin{compactitem}
  \item Cryptography is only useful if access to the keys and passwords is restricted.
  \item Logical restrictions still require physical restrictions to the system where the logical restrictions are maintained.
\end{compactitem}
\medskip

For large institutions, major difficulties are
\begin{compactitem}
 \item maintaining consistent and well-understood policies for these aspects, e.g.,
  \begin{compactitem}
   \item employees may ignore a policy they do not understand or disagree with,
  \end{compactitem} 
 \item avoiding unrealistic policies that are circumvented in practice, e.g.,
  \begin{compactitem}
   \item employees start writing down their passwords if the password policies are too strict,
   \item employees start buzzing in anybody if the door access policies are too strict
  \end{compactitem}
 \item avoiding routines that become predictable and easy to exploit, e.g.,
  \begin{compactitem}
   \item any encryption scheme becomes vulnerable if it is used for a long time by many people
   \item widely sharing a well-documented security policy also helps an attacker who gets access to it
  \end{compactitem}
 \item training employees not to give up sensitive information due to social engineering
  \begin{compactitem}
   \item see \url{https://en.wikipedia.org/wiki/Social_engineering_(security)}) for a good overview
  \end{compactitem} 
 \item updating anti-malware software over a large computer network including multiple vendors and operating systems, company-issued phones, and home office computers.
\end{compactitem}

\section{Bug Bounty Programs}

The legal situation for security professionals and researchers or simply users who stumble on a vulnerability is not always clear.
As a rule of thumb, it is safest to consider illegal anything that a computer system that was not meant to allow.
Depending on the country, gaining any kind of access to a computer system in an unintended way (even registering on Facebook with a fake name) may technically violate laws.

Bug bounty programs are a way for companies to provide these people with a legal framework to find, report, and help fix security problems.
In many cases, this is the only way how these people can do anything about security problems without violating laws.
However, even then bug bounty programs contain a number of restrictions.

A good overview of the legal situation is given in \url{https://www.srz.com/images/content/1/4/v2/149562/The-Cybersecurity-Law-Report-Proactive-Steps-to-Prevent-Legal-Pi.pdf}.

\section{Common Criteria}

The Common Criteria for Information Technology Security Evaluation (usually called Common Criteria) provide a framework for structured but informal specifications of the security properties of a computer system.
It allows companies to claim certain security properties and certifiers to test these properties.

Several government agencies use the Common Criteria to regulate the specification and certification processes when they are required by law.
