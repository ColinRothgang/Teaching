\section{Approaches}

\emph{Formal methods} is the area of computer science that develops static methods for verifying the correctness of software and hardware.
This is different from testing in that correctness is \emph{guaranteed}---systems are proved to be correct for arbitrary input (instead of dynamically checking a selected set of inputs).
It is different from the static analysis methods presented in Ch.~\ref{sec:sd:static} in that the \emph{entire} correctness is established (instead of checking for the presence of certain common errors).

Three branches can be roughly distinguished and are discussed separately in the following chapters.

All branches have in common that they use formal systems that involve both programs and formulas.
The latter is used to write specifications, the former to write implementations.
The goal is typically to show that the latter satisfies the former.

The separation between them is not clear-cut, and there are a lot of sharing of fundamental concepts, cross-fertilization, and and attempts at system integration between them.

\textbf{Deductive verification} focuses on representing the entire program as a term and logically proving the correctness.
Correctness is stated as a formula that is proved using appropriate proof rules.
This requires (semi)-manually transliterating the original program into the logic or automatically generating executable code from a program developed inside the logic.

\textbf{Model checking} focuses on statically analyzing a program only as much as necessary to establish correctness.
Often this involves building an abstraction of the program (which is written in an existing programming language) as, e.g., a state machine by analyzing its control flow graph.
Formal systems are mostly used to describe the behavior of the model checker, and not necessarily as the internal data structure of the model checker.

\textbf{Program synthesis} focuses on transforming (called \emph{refining}) the specification step-by-step into an implementation in such a way that correctness is guaranteed by construction.
Contrary to the other two branches, synthesis is applied to the specification, not to an existing implementation.

\section{Correctness}

By using a formal systems that combines logic and programming, we can concisely define what it means for a program to be correct.

\subsection{Programs}

\begin{definition}
A \textbf{program} is a term $t$ in some context $\Gamma$.

The \textbf{semantics} of $t$ consists of
\begin{compactitem}
 \item the term $t'$ such that $\evobj{}{\Gamma}{t}{t'}$,
 \item the mapping of initial states $s$ to transitions $(s,o,s')$ where $o$ is the sequence of operations triggered while evaluating $t$ and $s'$ is the resulting state.
\end{compactitem}

$t$ is called \textbf{pure} if
\begin{compactitem}
 \item $t'$ does not depend on the initial state $s$ (independence from the environment)
 \item $s=s'$ (no side effects)
\end{compactitem}
\end{definition}

Here $\Gamma$ contains all the
\begin{compactitem}
 \item declarations that are part of the program: classes, functions, global variables etc.
 \item the mathematical formalizations that are needed to write the specification: axioms, definitions, theorems, etc.
\end{compactitem}

The term $t$ is the initial function call that starts the program.
Often this a special constant named $main$ (possibly applied to arguments) defined somewhere in $\Gamma$.
Moreover, often $main$ is required to have $List[\String]\to \Int$ where the inputs are the command line arguments and the output is the exit code.
Then $t$ is of the form $main([arg_1,\ldots,arg_n]):\Int$.

In imperative languages, the semantics usually emphasizes the side effects of $t$ (especially the I/O operations that are observable by the user such as reading and printing).
The evaluation result $t'$ is negligible.

In functional languages, the semantics usually emphasizes the term $t'$, often restricting attention entirely to pure programs.

\subsection{Specifications}

The specification of a program is a set of axioms that characterize the intended semantics.

\begin{definition}[Specification]
Consider a program $t:A$ in some context $\Gamma$.

A \textbf{specification} for it consists of
\begin{compactitem}
\item a \textbf{precondition}: a formula in context $\Gamma$,
\item a \textbf{postcondition}: a formula in context $\Gamma$.
\end{compactitem}
\end{definition}

We think of the precondition as a property of the environment that we assume holds before evaluating $t$.
We think of the postcondition as a property of the environment that we guarantee will hold after evaluating $t$.

\subsection{Correctness}

For deterministic programs, we can define correctness as follows:

\begin{definition}[Correctness]
Consider a program $t$ in context $\Gamma$ and a specification with precondition $P$ and postcondition $Q$.

With respect to the specification, $t$ is called 
\begin{compactitem}
\item \textbf{partially correct} if $\iscons{}{\Gamma}{}{P\impl \abox{t}{Q}}$, i.e., if the precondition holds and $t$ terminates, then the postcondition holds afterwards.
\item \textbf{terminating} if $\iscons{}{\Gamma}{}{P\impl \adia{t}{\true}}$, i.e., if the precondition holds, then $t$ terminates.
\item \textbf{totally correct} if $\iscons{}{\Gamma}{}{P\impl \adia{t}{Q}}$, i.e., if the precondition holds, then $t$ terminates and the postcondition holds afterwards.
\end{compactitem}
Equivalently, total correctness is the conjunction of partial correctness and termination.
\end{definition}

The triple of $P$, $Q$, and $t$ is also called a Hoare-triple after their inventor. %\cite{hoarelogic}
In the context of Hoare-triples, partial correctness is often written as $\{P\}t\{Q\}$.

\subsection{Undecidability}

We assume a reasonably expressive logic, i.e., it should include at least first-order logic.
Then given a program and a specification, each of the following is undecidable:
\begin{compactitem}
 \item total correctness
 \item partial correctness
 \item termination
\end{compactitem}
Even termination without a precondition (i.e., $\adia{t}{\true}$) is undecidable.

For any problem that involves termination, that follows immediately from the undecidability of the halting problem.
For any problem that involves a postcondition, that follows immediately from the undecidability of first-order logic theorems.

\section{Annotating Programs to Aid Formal Methods}

Due to undecidability, we cannot expect any automated formal correctness arguments.
In order to aid automation, we have to attach additional information to our programs that guides correctness arguments.
These can be seen as stepping stones for correctness proofs.

Several such annotations are common.

\begin{example}[Running Example]\label{ex:sd:sqrt}
The following function computes the approximate square root:
\begin{acode}
\afun[\Int]{sqrt}{n:\Int}{
rt := 0\\
sq := 0\\
\awhile{sq \leq n}{
  rt := rt + 1\\
  sq := sq + 2\cdot rt + 1
}\\
\areturn{rt-1}
}
\end{acode}
Its intended meaning and correctness are non-obvious, even for humans.
\end{example}

\subsection{Function Specifications}

\subsubsection{Definition}

We can attach pre- and postconditions to function declarations.
These allow reasoning about the result of a function call in one big step without inspecting the body of the function.

\begin{definition}[Function Specification]\label{def:fm:funspec}
Consider a function declaration
 \[\afunI[A]{f}{x_1:A_1,\ldots, x_n:A_n}{t}\]
that is part of a fixed context $\Gamma$.

Let $\gamma=\aval{x_1}{A_1}{},\ldots,\aval{x_n}{A_n}{}$.

A \textbf{specification} for $f$ consists of
\begin{compactitem}
 \item precondition: a formula $P(x_1,\ldots,x_n)$ in context $\Gamma,\gamma$,
 \item postcondition: a formula $Q(x_1,\ldots,x_n,x)$ in context $\Gamma,\gamma,x:A$.
\end{compactitem}

Relative to the specification, $f$ is partially correct/terminating/totally correct if the program $x:=f(x_1,\ldots,x_n)$ is.
\end{definition}

The extra variables $x_i$ and $x$ represent the input and output of $f$.

Very often, we are interested in verifying a function by itself, as a modular component.
In that case, we use a specification of $f$ in the minimal (ideally: empty) context $\Gamma$.
Then correctness results can be reused in any larger context that comes up during program execution.

\begin{example}[Continuing Ex.~\ref{ex:sd:sqrt}]\label{ex:sd:sqrt2}
We specify $sqrt$ as follows:

The precondition $P(n)$ is $n\geq 0$.
The postcondition $Q(n,x)$ is $x^2\leq n < (x+1)^2$.
\end{example}

\subsubsection{Application}

Consider a partially correct function with $f$, $P$ and $Q(x)$, as in Def.~\ref{def:fm:funspec}.
Moreover, assume that the body of $f$ is pure.

Then the following proof rule is sound:

\[\rul{\iscons{}{\Gamma}{}{P(t_1,\ldots,t_n)} \tb \iscons{}{\Gamma,\aval{x}{A}{}}{}{}{Q(t_1,\ldots,t_n,x)\impl F}}{\iscons{}{\Gamma}{}{\abox{x:=f(t_1,\ldots,t_n)}{F}}}\]

This can be read as follows: to call $f$, we
\begin{compactenum}
 \item establish the precondition of $f$,
 \item assume the postcondition of $f$ holds about the result, and
 \item continue the correctness proof
\end{compactenum}

If $f$ is totally correct, we obtain a corresponding rule for the diamond operator.


\subsection{Loop Invariants}

\subsubsection{Definition}

Similar to function specification allowing us to reason about a function call in one big step, a loop invariant allows reasoning about a while-loop in one big step.
Loop invariants are critical to prove the partial correctness of while-loops.

\begin{definition}[Loop Invariant]
Consider a context $\Gamma$ and a while-loop $\awhileI{C}{t}$ in which $C$ is pure.

A \textbf{loop invariant} is a formula $F$ such that
 \[\iscons{}{\Gamma}{}{(I\wedge C)\impl \abox{t}{I}}\]
\end{definition}

Intuitively, a loop invariant is a formula whose truth is preserved by executing the body of the loop.

\begin{example}[Continuing Ex.~\ref{ex:sd:sqrt2}]\label{ex:sd:sqrt3}
The following are loop invariants for the while-loop in $sqrt$:
\begin{compactitem}
 \item $\true$: technically correct, but useless
 \item $\false$: technically correct, but useless (also, does not hold at the beginning of the loop)
 \item $rt>0 \wedge sq > 0$: better but will not help prove the postcondition
 \item $rt^2==sq$: the one that captures the idea of the loop and is useful in practice
\end{compactitem}

Note how it is basically impossible to figure out the loop invariant without first understanding how the algorithm works.
We cannot expect being able to automate this process.

So put $I=rt^2==sq$.
We can now prove that
\[\iscons{}{\aval{rt}{\Int}{};\;\aval{sq}{\Int}{}}{}{(I\wedge \neg sq\leq n)\impl \abox{rt := rt + 1;\; sq := sq + 2\cdot rt + 1}{I}}\]
i.e., $I$ is a loop invariant, and
\end{example}

\subsubsection{Application}

If we have a loop invariant $I$, the following proof rule is sound:

\[\rul{\iscons{}{\Gamma}{}{I} \tb \iscons{}{\Gamma}{}{I\wedge \neg C \impl F}}
      {\iscons{}{\Gamma}{}{\abox{\awhileI{C}{I}}{F}}}\]

We can read this as follows:
to execute a while-loop,
\begin{compactenum}
 \item establish the loop-invariant $I$,
 \item execute the body loop until $C$ becomes false (which always preserves $I$),
 \item assume $I$ and $\neg C$, and
 \item continue the proof.
\end{compactenum}

\begin{example}[Continuing Ex.~\ref{ex:sd:sqrt3}]\label{ex:sd:sqrt4}
To apply the loop invariant $I$, we first have to prove that
\[\iscons{}{\aval{n}{\Int}{},\;\aval{rt}{\Int}{0},\;\aval{sq}{\Int}{0}}{}{rt^2==sq}\]
i.e., $I$ holds at the beginning of the loop.
That is easy.

Then we have to prove that the loop invariant and the negative loop condition allow proving the postcondition about the return value $rt-1$:
\[\iscons{}{\aval{n}{\Int}{},\;\aval{rt}{\Int}{};\;\aval{sq}{\Int}{}}{}{rt^2==sq\wedge \neg sq\leq n \impl (rt-1)^2\leq n < (rt-1+1)^2}\]
The proof of $n<rt^2$ is easy.

But we cannot easily prove $(rt-1)^2\leq n$.
We know it is true: when the while-loop terminates, not only is $sq>n$ but it must be the smallest square number above $n$.
(Because each iteration of the loop increments $rt$ by $1$ only, we cannot skip a square number.)
But that knowledge is not captured by the loop invariant.

Thus, we have to go back and prove a strengthened loop invariant.
For example,
\[rt^2==sq \wedge \neg \aex[\Int]{k}{n < k^2 < rt^2}\]
is also a loop invariant (which is trivial to prove) and is strong enough to prove the postcondition.
\end{example}

\subsection{Class Invariants}

For simplicity, we will only work with a very simple special case of classes.
We assume a class declaration is of the form
\[\aclassI{C}{x:A}{}{\ldots,\aval{f_i}{B_i\to B'_i}{t_i},\ldots}\]
where
\begin{compactitem}
 \item $x$ is the only constructor argument and the only mutable fields of $C$,
 \item $x$ is private to the class,
 \item all other fields of $C$ are methods $f_i$ that take exactly one argument.
\end{compactitem}
We do allow the characteristic property of object-orientatiton: the bodies $t_i$ of the methods may use and assign to the private variables $x$.

No fundamental conceptual difficulties arise when generalizing to arbitrary classes, but, depending on the details, the complexity of giving class invariants can increase substantially.

\subsubsection{Definition}

A class invariant is similar to a loop invariant: it is a formula whose truth is preserved by operations.
In a loop invariant, the body of the loop is the only operation.
In a class, any method call is a possible operation.

\begin{definition}[Class Invariant]
Consider a class declaration $C$ as above that is part of a context $\Gamma$.

A \textbf{class invariant} is a formula $I(c)$ in context $\Gamma,c:C$ such that
\[\iscons{}{\Gamma,\aval{a}{A}{}}{}{I(\anew{C}{a})}\]
and for every method $f:B\to B'$
\[\iscons{}{\Gamma,\aval{c}{C}{},\aval{b}{B}{}}{}{I(c)\impl \abox{c.f(b)}{I(c)}}\]
\end{definition}

Intuitively, $I(o)$ holds for all new instances $c$ and cannot be invalidated by any method calls on $c$.

\begin{example}{Size of stack}
The following implementation of stacks allows for constant-time computation of the size of the stack:
\begin{acode}
\aclass{Stack[A]}{}{}{
 \akey{private} elements := Nil[A] \acomment{the immutable linked list backing the stack, initially empty}\\
 \akey{private} size := 0 \\
 \afunI[\N]{getSize}{}{size}\\
 \afun[\Unit]{push[A]}{x:A}{
    elements := cons(x,elements)\\
    size:=size+1
 }\\
 \afun[{Option[A]}]{pop[A]}{}{
    \aifelse{elements == []}{
      None
    }{
      e := elements.head\\
      elements := elements.tail\\
      size :=size-1\\
      Some(e)
    }
 }
}
\end{acode}

An important class invariant for it is $length(elements)==size$.
\end{example}

Similar class invariants come up every time a class store the same data in multiple ways for efficiency or redundancy.

Another important use case are validity conditions about auxiliary data that is built over the lifetime of a class:

\begin{example}[Formal System Case Study]
The parser class from the formal system example satisfies the class invariant $0\leq pos\leq length(input)$.

The interpreter class satisfies the class invariant that the environment contains only well-typed terms.
\end{example}

\subsubsection{Application}

Because the $\anew{C}{x}$ operator is the only way to every obtain a new instance of $C$, a class invariant must be true for all instances at all times.

Formally, if we have a class invariant $I(c)$ for a class $C$, the following proof rule is sound:

\[\rul{}
      {\iscons{}{\Gamma,\aval{c}{C}{}}{}{\square I(c)}}\]


\subsection{Termination Orderings}

Partial correctness and termination usually require very different arguments.
Therefore, they are often treated separately.
The annotations described above help with proving partial correctness.
Termination orderings help with proving termination.

\subsubsection{Definition}

We consider the two most important sources of possible non-termination: while-loops and recursion.

\paragraph{Loops}
A termination ordering for a while-loop is a term of type $\N$ that strictly decreases with every iteration of the loop.
Because natural numbers can only strictly decrease finitely many times, the existence of a termination ordering guarantees the termination of the overall loop.

We assume that a specification of the natural numbers $\N$ with the usual ordering $<$ is part of all contexts $\Gamma$.

\begin{definition}[Termination Ordering for a while-loop]
Consider a context $\Gamma$ and a while-loop $\awhileI{C}{B}$ in which $C$ is pure.

A \textbf{termination ordering} is a pure term $\oftype{}{\Gamma}{t}{\N}$ such that
 \[\iscons{}{\Gamma,\aval{before}{\N}{t}}{}{C\impl \abox{B}{\{\aval{after}{\N}{t}; after<before\}}}\]
\end{definition}

Here $before$ and $after$ contain the values of $t$ before and after evaluating the body $B$.
And $after<before$ expresses that $t$ strictly decreases.

\begin{example}[Continuing Ex.~\ref{ex:sd:sqrt}]\label{ex:sd:sqrt5}
A termination ordering for the while-loop is given by $n-sq$.
\end{example}

\paragraph{Recursive Functions}

A termination ordering for a recursive function expresses that the size of the arguments of the function decreases with every recursive call.

\begin{definition}[Termination Ordering for a recursive function]
Consider a context $\Gamma$ and a pure function $\afunI[A]{f}{x_1:A_1,\ldots,x_n:A_n}{B}$.

A \textbf{termination ordering} is a pure term $t(x_1,\ldots,x_n)$ in context $\Gamma,x_1:A_n,\ldots,x_n:A_n$ such that for every recursive call $f(a_1,\ldots,a_n)$ inside $t$, we have
 \[\iscons{}{\Gamma,x_1:A_n,\ldots,x_n:A_n}{}{t(a_1,\ldots,a_n)<t(x_1,\ldots,x_n)}\]
\end{definition}

Here $t(x_1,\ldots,x_n)$ corresponds to the $before$-value---the function arguments in the current call---and $t(a_1,\ldots,a_b)$ correspond to the after-value---the function arguments in the next call of $f$.

\subsubsection{Application}

If there is a termination ordering for a while-loop, the loop is guaranteed to terminate.

If there is a termination ordering for a recursive function $f$, every function call $f(v_1,\ldots,v_n)$ is guaranteed to terminate.

\subsubsection{Generalization}

In practice, it is not always easy to give a termination ordering $t:\N$.
However, we can generalize the argument as follows.

Let $(A,<)$ be a strict ordering.
Moreover, assume a $<$-preserving function $f:A\to \N$, i.e., a function $f$ such $a<a'$ implies $f(a)<f(a')$.

Then any term $t:A$ can be used as a termination ordering.

\paragraph{Subterm-Ordering}
The most important special case is the subterm ordering on inductive data types.

Let $A$ be an inductive data type.
We define a strict ordering $<$ on $A$ by: $s<t$ iff $s$ is a proper subterm of $t$.

Then the function $f:A\to \N$ that maps every term to the height of its syntax tree preserves $<$.

The subterm ordering is extremely helpful to prove the termination of recursive functions on inductive data types: almost all the time such functions recurse only into subterms.
The termination of any such function follows immediately.